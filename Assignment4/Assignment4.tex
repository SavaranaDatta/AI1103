  
\documentclass[journal,12pt,twocolumn]{IEEEtran}

\usepackage{setspace}
\usepackage{gensymb}
\singlespacing
\usepackage[cmex10]{amsmath}

\usepackage{amsthm}

\usepackage{mathrsfs}
\usepackage{txfonts}
\usepackage{stfloats}
\usepackage{bm}
\usepackage{cite}
\usepackage{cases}
\usepackage{subfig}

\usepackage{longtable}
\usepackage{multirow}

\usepackage{enumerate}
\usepackage{mathtools}
\usepackage{steinmetz}
\usepackage{tikz}
\usepackage{circuitikz}
\usepackage{verbatim}
\usepackage{tfrupee}
\usepackage[breaklinks=true]{hyperref}
\usepackage{graphicx}
\usepackage{tkz-euclide}

\usetikzlibrary{calc,math}
\usepackage{listings}
    \usepackage{color}                                            %%
    \usepackage{array}                                            %%
    \usepackage{longtable}                                        %%
    \usepackage{calc}                                             %%
    \usepackage{multirow}                                         %%
    \usepackage{hhline}                                           %%
    \usepackage{ifthen}                                           %%
    \usepackage{lscape}     
\usepackage{multicol}
\usepackage{chngcntr}

\DeclareMathOperator*{\Res}{Res}

\renewcommand\thesection{\arabic{section}}
\renewcommand\thesubsection{\thesection.\arabic{subsection}}
\renewcommand\thesubsubsection{\thesubsection.\arabic{subsubsection}}

\renewcommand\thesectiondis{\arabic{section}}
\renewcommand\thesubsectiondis{\thesectiondis.\arabic{subsection}}
\renewcommand\thesubsubsectiondis{\thesubsectiondis.\arabic{subsubsection}}


\hyphenation{op-tical net-works semi-conduc-tor}
\def\inputGnumericTable{}                                 %%

\lstset{
%language=C,
frame=single, 
breaklines=true,
columns=fullflexible
}
\begin{document}

\newcommand{\BEQA}{\begin{eqnarray}}
\newcommand{\EEQA}{\end{eqnarray}}
\newcommand{\define}{\stackrel{\triangle}{=}}
\bibliographystyle{IEEEtran}
\raggedbottom
\setlength{\parindent}{0pt}
\providecommand{\mbf}{\mathbf}
\providecommand{\pr}[1]{\ensuremath{\Pr\left(#1\right)}}
\providecommand{\qfunc}[1]{\ensuremath{Q\left(#1\right)}}
\providecommand{\sbrak}[1]{\ensuremath{{}\left[#1\right]}}
\providecommand{\lsbrak}[1]{\ensuremath{{}\left[#1\right.}}
\providecommand{\rsbrak}[1]{\ensuremath{{}\left.#1\right]}}
\providecommand{\brak}[1]{\ensuremath{\left(#1\right)}}
\providecommand{\lbrak}[1]{\ensuremath{\left(#1\right.}}
\providecommand{\rbrak}[1]{\ensuremath{\left.#1\right)}}
\providecommand{\cbrak}[1]{\ensuremath{\left\{#1\right\}}}
\providecommand{\lcbrak}[1]{\ensuremath{\left\{#1\right.}}
\providecommand{\rcbrak}[1]{\ensuremath{\left.#1\right\}}}
\theoremstyle{remark}
\newtheorem{rem}{Remark}
\newcommand{\sgn}{\mathop{\mathrm{sgn}}}
\providecommand{\abs}[1]{\vert#1\vert}
\providecommand{\res}[1]{\Res\displaylimits_{#1}} 
\providecommand{\norm}[1]{\lVert#1\rVert}
%\providecommand{\norm}[1]{\lVert#1\rVert}
\providecommand{\mtx}[1]{\mathbf{#1}}
\providecommand{\mean}[1]{E[ #1 ]}
\providecommand{\fourier}{\overset{\mathcal{F}}{ \rightleftharpoons}}
%\providecommand{\hilbert}{\overset{\mathcal{H}}{ \rightleftharpoons}}
\providecommand{\system}{\overset{\mathcal{H}}{ \longleftrightarrow}}
	%\newcommand{\solution}[2]{\textbf{Solution:}{#1}}
\newcommand{\solution}{\noindent \textbf{Solution: }}
\newcommand{\cosec}{\,\text{cosec}\,}
\providecommand{\dec}[2]{\ensuremath{\overset{#1}{\underset{#2}{\gtrless}}}}
\newcommand{\myvec}[1]{\ensuremath{\begin{pmatrix}#1\end{pmatrix}}}
\newcommand{\mydet}[1]{\ensuremath{\begin{vmatrix}#1\end{vmatrix}}}
\numberwithin{equation}{subsection}
\makeatletter
\@addtoreset{figure}{problem}
\makeatother
\let\StandardTheFigure\thefigure
\let\vec\mathbf
\renewcommand{\thefigure}{\theproblem}
\def\putbox#1#2#3{\makebox[0in][l]{\makebox[#1][l]{}\raisebox{\baselineskip}[0in][0in]{\raisebox{#2}[0in][0in]{#3}}}}
     \def\rightbox#1{\makebox[0in][r]{#1}}
     \def\centbox#1{\makebox[0in]{#1}}
     \def\topbox#1{\raisebox{-\baselineskip}[0in][0in]{#1}}
     \def\midbox#1{\raisebox{-0.5\baselineskip}[0in][0in]{#1}}
\vspace{3cm}
\title{Assignment 4}
\author{SAVARANA DATTA - AI20BTECH11008}
\maketitle
\newpage
\bigskip
\renewcommand{\thefigure}{\theenumi}
\renewcommand{\thetable}{\theenumi}
Download all python codes from 
\begin{lstlisting}
https://github.com/SavaranaDatta/AI1103/blob/main/Assignment4/codes/Assignment4.py
\end{lstlisting}
%
and latex-tikz codes from 
%
\begin{lstlisting}
https://github.com/SavaranaDatta/AI1103/blob/main/Assignment4/Assignment4.tex
\end{lstlisting}
\section{Problem(CS 2010 Q.26)}
Consider a company that assembles computers.The probability of a faulty assembly of any computer is p.The company therefore subjects each computer to testing process.This testing process gives the correct result for any computer with a probability of q.What is the probability of a computer being declared faulty?
\begin{enumerate}[(A)]
\item pq+(1-p)(1-q)
\item (1-q)p
\item (1-p)q
\item pq
\end{enumerate}
\section{solution(CS 2010 Q.26)}
Let $X_i \in \cbrak{0,1}$ where $\pr{X_1 = 1}$ represents the computer is faulty before testing, $\pr{X_2 = 1}$ represents the testing process gives the correct result.
\begin{table}[h]
\centering 
\caption{}
\begin{tabular}{|c|c|c|}
\hline
           & $X_1 = 0$  & $X_1 = 1$\\
\hline
$X_2 = 0$  & (1-p)(1-q) & (1-q)p \\
\hline
$X_2 = 1$  &  (1-p)q    &  pq \\
\hline
\end{tabular}
\label{table:}
\end{table}
 
Table \ref{table:} represents the probabilities of all possible cases.The probability of a computer being declared as faulty is 
\begin{align}
\tag{1.1}
     &= \pr{(X_2 = 1)(X_1 = 1)}+\pr{(X_2 = 0)(X_1 = 0)}\\
\tag{1.2}
     &= pq+(1-p)(1-q) 
\end{align}
The required option is (A).
\begin{figure}[H]
\centering
\label{plot1}
\includegraphics[width=\columnwidth]{download.png}
\caption{Theoretical vs simulation for $p=0.272,q=0.688$}
\end{figure}
\end{document}