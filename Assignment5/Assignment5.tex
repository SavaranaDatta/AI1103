  
\documentclass[journal,12pt,twocolumn]{IEEEtran}

\usepackage{setspace}
\usepackage{gensymb}
\singlespacing
\usepackage[cmex10]{amsmath}

\usepackage{amsthm}

\usepackage{mathrsfs}
\usepackage{txfonts}
\usepackage{stfloats}
\usepackage{bm}
\usepackage{cite}
\usepackage{cases}
\usepackage{subfig}

\usepackage{longtable}
\usepackage{multirow}

\usepackage{enumerate}
\usepackage{mathtools}
\usepackage{steinmetz}
\usepackage{tikz}
\usepackage{circuitikz}
\usepackage{verbatim}
\usepackage{tfrupee}
\usepackage[breaklinks=true]{hyperref}
\usepackage{graphicx}
\usepackage{tkz-euclide}

\usetikzlibrary{calc,math}
\usepackage{listings}
    \usepackage{color}                                            %%
    \usepackage{array}                                            %%
    \usepackage{longtable}                                        %%
    \usepackage{calc}                                             %%
    \usepackage{multirow}                                         %%
    \usepackage{hhline}                                           %%
    \usepackage{ifthen}                                           %%
    \usepackage{lscape}     
\usepackage{multicol}
\usepackage{chngcntr}

\DeclareMathOperator*{\Res}{Res}

\renewcommand\thesection{\arabic{section}}
\renewcommand\thesubsection{\thesection.\arabic{subsection}}
\renewcommand\thesubsubsection{\thesubsection.\arabic{subsubsection}}

\renewcommand\thesectiondis{\arabic{section}}
\renewcommand\thesubsectiondis{\thesectiondis.\arabic{subsection}}
\renewcommand\thesubsubsectiondis{\thesubsectiondis.\arabic{subsubsection}}


\hyphenation{op-tical net-works semi-conduc-tor}
\def\inputGnumericTable{}                                 %%

\lstset{
%language=C,
frame=single, 
breaklines=true,
columns=fullflexible
}
\begin{document}

\newcommand{\BEQA}{\begin{eqnarray}}
\newcommand{\EEQA}{\end{eqnarray}}
\newcommand{\define}{\stackrel{\triangle}{=}}
\bibliographystyle{IEEEtran}
\raggedbottom
\setlength{\parindent}{0pt}
\providecommand{\mbf}{\mathbf}
\providecommand{\pr}[1]{\ensuremath{\Pr\left(#1\right)}}
\providecommand{\qfunc}[1]{\ensuremath{Q\left(#1\right)}}
\providecommand{\sbrak}[1]{\ensuremath{{}\left[#1\right]}}
\providecommand{\lsbrak}[1]{\ensuremath{{}\left[#1\right.}}
\providecommand{\rsbrak}[1]{\ensuremath{{}\left.#1\right]}}
\providecommand{\brak}[1]{\ensuremath{\left(#1\right)}}
\providecommand{\lbrak}[1]{\ensuremath{\left(#1\right.}}
\providecommand{\rbrak}[1]{\ensuremath{\left.#1\right)}}
\providecommand{\cbrak}[1]{\ensuremath{\left\{#1\right\}}}
\providecommand{\lcbrak}[1]{\ensuremath{\left\{#1\right.}}
\providecommand{\rcbrak}[1]{\ensuremath{\left.#1\right\}}}
\theoremstyle{remark}
\newtheorem{rem}{Remark}
\newcommand{\sgn}{\mathop{\mathrm{sgn}}}
\providecommand{\abs}[1]{\vert#1\vert}
\providecommand{\res}[1]{\Res\displaylimits_{#1}} 
\providecommand{\norm}[1]{\lVert#1\rVert}
%\providecommand{\norm}[1]{\lVert#1\rVert}
\providecommand{\mtx}[1]{\mathbf{#1}}
\providecommand{\mean}[1]{E[ #1 ]}
\providecommand{\fourier}{\overset{\mathcal{F}}{ \rightleftharpoons}}
%\providecommand{\hilbert}{\overset{\mathcal{H}}{ \rightleftharpoons}}
\providecommand{\system}{\overset{\mathcal{H}}{ \longleftrightarrow}}
	%\newcommand{\solution}[2]{\textbf{Solution:}{#1}}
\newcommand{\solution}{\noindent \textbf{Solution: }}
\newcommand{\cosec}{\,\text{cosec}\,}
\providecommand{\dec}[2]{\ensuremath{\overset{#1}{\underset{#2}{\gtrless}}}}
\newcommand{\myvec}[1]{\ensuremath{\begin{pmatrix}#1\end{pmatrix}}}
\newcommand{\mydet}[1]{\ensuremath{\begin{vmatrix}#1\end{vmatrix}}}
\numberwithin{equation}{subsection}
\makeatletter
\@addtoreset{figure}{problem}
\makeatother
\let\StandardTheFigure\thefigure
\let\vec\mathbf
\renewcommand{\thefigure}{\theproblem}
\def\putbox#1#2#3{\makebox[0in][l]{\makebox[#1][l]{}\raisebox{\baselineskip}[0in][0in]{\raisebox{#2}[0in][0in]{#3}}}}
     \def\rightbox#1{\makebox[0in][r]{#1}}
     \def\centbox#1{\makebox[0in]{#1}}
     \def\topbox#1{\raisebox{-\baselineskip}[0in][0in]{#1}}
     \def\midbox#1{\raisebox{-0.5\baselineskip}[0in][0in]{#1}}
\vspace{3cm}
\title{Assignment 5}
\author{SAVARANA DATTA - AI20BTECH11008}
\maketitle
\newpage
\bigskip
\renewcommand{\thefigure}{\theenumi}
\renewcommand{\thetable}{\theenumi}
Download latex-tikz codes from 
%
\begin{lstlisting}
https://github.com/SavaranaDatta/AI1103/blob/main/Assignment5/Assignment5.tex
\end{lstlisting}
\section*{Problem(UGC 2018(Dec math set-a), Q.111)}
Let $X_{1},X_{2},X_{3},..,X_{n}$ be independent random variables follow a common continuous distribution \textbf{F}, which is symmetric about 0. For i=1,2,3,..n, define 
\begin{align}
\tag{1.1}
S_{i} = 
\begin{cases}
1 & if \hspace{0.2cm}X_{i}>0
\\
-1 & if\hspace{0.2cm} X_{i}<0 \hspace{0.2cm} and
\\
0 & if \hspace{0.2cm}X_{i}=0
\end{cases}
\label{pdf}
\end{align}

$R_{i}$=rank of $|X_{i}|$ in the set\{$|X_{1}|,|X_{2}|,..,|X_{n}|$\}.Which of the following statements are correct?
\begin{enumerate}[(A)]
\item $S_{1},S_{2},..,S_{n}$ are independent and identically distributed.
\item $R_{1},R_{2},..,R_{n}$ are independent and identically distributed.
\item $S=\brak{S_{1},S_{2},..,S_{n}}$ and $R=\brak{ R_{1},R_{2},..,R_{n}}$ are independent.
\end{enumerate}
\section*{Solution(UGC 2018(Dec math set-a), Q.111)}
A sequence $\{X_{i}\}$ is an Independent and identical if and only if 
$$F_{X_{n}}(x)=F_{X_{k}}(x)$$
$\forall$ n,k,x and any subset of terms of the sequence is a set of mutually independent random variables.
Where F is the probability density function.
\subsection*{Option(A)}
As the probability distribution function of $\{X_{i}\}$ is symmetric about origin we can say that $$F_{S_{n}}(s)=F_{S_{k}}(s)\hspace{0.5cm}\forall s,k,n$$
Any subset of terms of sequence $\{S_{i}\}$ is a set of mutually independent random variables. So, the sequence $\{S_{i}\}$ is independent and identical.
\subsection*{Option (B)}
Ranking of a sequence depend on every elements of the sequence.As $\{R_{i}\}$ is a ranking function of $\{X_{i}\}$, we can say that $\{R_{i}\}$ is not an independent function. Hence, it is not independent and identical. 
\subsection*{Option (C)}
As the $i^{th}$ element of sequence R depends only on $X_{i}$, we can say that sequence S and R are independent.

Answer:A,C
\end{document}
