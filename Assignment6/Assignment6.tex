\documentclass[journal,12pt,twocolumn]{IEEEtran}

\usepackage{setspace}
\usepackage{gensymb}
\singlespacing
\usepackage[cmex10]{amsmath}
\usepackage{cancel}
\usepackage{amsthm}

\usepackage{paralist}

\usepackage{mathrsfs}
\usepackage{txfonts}
\usepackage{stfloats}
\usepackage{bm}
\usepackage{cite}
\usepackage{cases}
\usepackage{subfig}

\usepackage{longtable}
\usepackage{multirow}
\usepackage{enumitem}
\usepackage{mathtools}
\usepackage{steinmetz}
\usepackage{tikz}
\usepackage{circuitikz}
\usepackage{verbatim}
\usepackage{tfrupee}
\usepackage[breaklinks=true]{hyperref}
\usepackage{graphicx}
\usepackage{tkz-euclide}

\usetikzlibrary{calc,math}
\usepackage{listings}
    \usepackage{color}                                            %%
    \usepackage{array}                                            %%
    \usepackage{longtable}                                        %%
    \usepackage{calc}                                             %%
    \usepackage{multirow}                                         %%
    \usepackage{hhline}                                           %%
    \usepackage{ifthen}                                           %%
    \usepackage{lscape}     
\usepackage{multicol}
\usepackage{chngcntr}

\DeclareMathOperator*{\Res}{Res}
\usepackage{romannum}
\renewcommand\thesection{\arabic{section}}
\renewcommand\thesubsection{\thesection.\arabic{subsection}}
\renewcommand\thesubsubsection{\thesubsection.\arabic{subsubsection}}

\renewcommand\thesectiondis{\arabic{section}}
\renewcommand\thesubsectiondis{\thesectiondis.\arabic{subsection}}
\renewcommand\thesubsubsectiondis{\thesubsectiondis.\arabic{sub subsection}}
\newtheorem{theorem}{Theorem}
\newtheorem{claim}[theorem]{Claim}
\newtheorem{proposition}[theorem]{Proposition}
\newtheorem{lemma}[theorem]{Lemma}
\newtheorem{corollary}[theorem]{Corollary}
\newtheorem{conjecture}[theorem]{Conjecture}
\newtheorem*{observation}{Observation}
\newtheorem*{example}{Example}
\newtheorem*{remark}{Remark}

\hyphenation{optical networks semiconduc-tor}
\def\inputGnumericTable{}                                 %%

\lstset{
%language=C,
frame=single, 
breaklines=true,
columns=fullflexible
}
\date{March 2021}

\begin{document}
\theoremstyle{definition}
\newtheorem{definition}{Definition}[section]

\newcommand{\BEQA}{\begin{eqnarray}}
\newcommand{\EEQA}{\end{eqnarray}}
\newcommand{\define}{\stackrel{\triangle}{=}}
\bibliographystyle{IEEEtran}
\raggedbottom
\setlength{\parindent}{0pt}
\providecommand{\mbf}{\mathbf}
\providecommand{\pr}[1]{\ensuremath{\Pr\left(#1\right)}}
\providecommand{\qfunc}[1]{\ensuremath{Q\left(#1\right)}}
\providecommand{\fn}[1]{\ensuremath{f\left(#1\right)}}
\providecommand{\e}[1]{\ensuremath{E\left(#1\right)}}
\providecommand{\sbrak}[1]{\ensuremath{{}\left[#1\right]}}
\providecommand{\lsbrak}[1]{\ensuremath{{}\left[#1\right.}}
\providecommand{\rsbrak}[1]{\ensuremath{{}\left.#1\right]}}
\providecommand{\brak}[1]{\ensuremath{\left(#1\right)}}
\providecommand{\lbrak}[1]{\ensuremath{\left(#1\right.}}
\providecommand{\rbrak}[1]{\ensuremath{\left.#1\right)}}
\providecommand{\cbrak}[1]{\ensuremath{\left\{#1\right\}}}
\providecommand{\lcbrak}[1]{\ensuremath{\left\{#1\right.}}
\providecommand{\rcbrak}[1]{\ensuremath{\left.#1\right\}}}
\theoremstyle{remark}
\newtheorem{rem}{Remark}
\newcommand{\sgn}{\mathop{\mathrm{sgn}}}
\providecommand{\abs}[1]{\vert#1\vert}
\providecommand{\res}[1]{\Res\displaylimits_{#1}} 
\providecommand{\norm}[1]{\lVert#1\rVert}
%\providecommand{\norm}[1]{\lVert#1\rVert}
\providecommand{\mtx}[1]{\mathbf{#1}}
\providecommand{\mean}[1]{E[ #1 ]}
\providecommand{\fourier}{\overset{\mathcal{F}}{ \rightleftharpoons}}
%\providecommand{\hilbert}{\overset{\mathcal{H}}{ \rightleftharpoons}}
\providecommand{\system}{\overset{\mathcal{H}}{ \longleftrightarrow}}
	%\newcommand{\solution}[2]{\textbf{Solution:}{#1}}
\newcommand{\solution}{\noindent \textbf{Solution: }}
\newcommand{\cosec}{\,\text{cosec}\,}
\providecommand{\dec}[2]{\ensuremath{\overset{#1}{\underset{#2}{\gtrless}}}}
\newcommand{\myvec}[1]{\ensuremath{\begin{pmatrix}#1\end{pmatrix}}}
\newcommand{\mydet}[1]{\ensuremath{\begin{vmatrix}#1\end{vmatrix}}}
\numberwithin{equation}{subsection}
\makeatletter
\vspace{3cm}
\title{Assignment 6}
\author{G SAVARANA DATTA REDDY - AI20BTECH11008}
\maketitle
\newpage
\bigskip
\renewcommand{\thetable}{\theenumi}
Download the latex code from 
\begin{lstlisting}
https://github.com/SavaranaDatta/AI1103/tree/main/Assignment_6.tex
\end{lstlisting}
\section{PROBLEM}
Let X$_{1}$ and X$_{2}$ be a random sample of size two
from a distribution with probability density
function
\begin{align}
    f_{\theta}(x) &= \theta \brak{\dfrac{1}{\sqrt{2\pi}}}e^{-\dfrac{1}{2} x^{2}} + \brak{1-\theta}\brak{\dfrac{1}{2}}e^{-\mid x\mid} \nonumber,
\end{align}
$-\infty<x<\infty$,\\
where  $\theta \in \cbrak{ 0,\dfrac{1}{2}, 1 }$. If the observed values
of X$_{1}$ and X$_{2}$ are 0 and 2, respectively, then
the maximum likelihood estimate of $\theta$ is
\begin{enumerate}
    \item 0 
    \item $\frac{1}{2}$
    \item 1
    \item not unique
\end{enumerate}
\section{SOLUTION}
\begin{definition}[Maximum Likelihood Estimation (MLE)]
Let $x_1, x_2, \ldots, x_n$
be observations from an independent and identically distributed random variables drawn from a Probability Distribution $f_0$
,where $f_0$ is known to be from a family of distributions f that depend on some parameters $\theta$.
\end{definition}
The goal of MLE is to maximize the likelihood function:
 \begin{align}
L(\theta)&= f(x_1,x_2, \ldots,x_n \; | \;\theta)\\
 &=f(x_1 \;|\; \theta)\times f(x_2 \; |\; \theta)\times....\times f(x_n \;|\;\theta)
\end{align}
\begin{lemma}\label{shit}
Let $f(x)$ be a differentiable function.If $f'(x)<0$ in the interval $x\in[a,b]$ then $f(x)$ attains its maximum value at $x=a$ in the interval $x\in[a,b].$
\end{lemma}
The likelihood function for given data is given by 
\begin{align}
    L(\theta \hspace{0.1cm}|\hspace{0.1cm}X_{1}=0,X_{2}=2)  &=f_{\theta}(X_{1}=0)\times f_{\theta}(X_{2}=2)\\
    &=\brak{\theta\brak{\dfrac{1}{\sqrt{2\pi}}-\frac{1}{2}}+\frac{1}{2}}^{2}e^{-2}
\end{align}
\begin{multline}
    \implies L'(\theta)=\\\brak{\sqrt{\frac{2}{\pi}}-1}e^{-2}\brak{\theta\brak{\dfrac{1}{\sqrt{2\pi}}-\frac{1}{2}}+\frac{1}{2}}
\end{multline}
$L'(\theta)<0$ in the interval [0,1]. So from Lemma $\ref{shit}$  we can say $L(\theta)$ is maximum at $\theta=0$.So the required option is (1).
%But $\theta \in \cbrak{ 0,\dfrac{1}{2}, 1 }$
%\begin{enumerate}
    %\item At $\theta=0$ \hspace{0.5cm} %$L(\theta=0)=\frac{1}{4}e^{-2}=0.0338$\\
    %\item At $\theta=1$ \hspace{0.5cm} %$L(\theta=1)=\frac{1}{2\pi}e^{-2}=0.0215$\\
    %\item At $\theta=\frac{1}{2}$ \hspace{0.2cm}
    %$L(\theta=\frac{1}{2})=\brak{\frac{1}{2\sqrt{2\p%i}}+\frac{1}{4}}^{2}e^{-2}=0.0273$
%\end{enumerate}
%$\therefore$ Required option is \textbf{1}.
\end{document}